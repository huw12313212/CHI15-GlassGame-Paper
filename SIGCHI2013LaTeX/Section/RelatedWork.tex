\section{Related Work}

Although there are lack of offcial research focusing on glass game design, we can still borrow some experience from the research of existing Game Design and Head-Mounted Device Control.

\subsection{Game Design}

There are tons of research focusing on design guideline or heuristic rule for traditional video games\cite{gameflow,criticalreview,chi04game,09game,02game,08game,07game}. Nevertheless, after the smart phone hitting the market, game design for mobile phones also became a hot research topic\cite{mobilegame,mobile06,mobile08,icec06}. Besides video and mobile game, there are only bits and pieces research on other specific platform. For example, Elena et al. focused on using their unique gaming hardware, ``Body Bug'', to explore body games\cite{bodygame}; Wetzel et al. explored the game design in augmented reaility\cite{argame}; Mueller et al. presented a set of guidelines for movement-based game design\cite{movegame}. Bertelsmeyer et al. provided an innovative wearable game and researched in how to consider the characteristics of today's wearable hardware when developing a game\cite{wearable}. Unfortunately, they didn't consider the smart glass as a gaming platform in their research. 
From the previous research on related field, although there are no research focusing on glass game design, some previous game design guidelines are platform independent and we can use it directly. For instance, Pinelle et al.\cite{videogame} developed game heuristics in 10 categories, most of them are platform independent. 
Such as ``Predictability'' issue , games should respond to users' actions in a predictable manner. And ``Game status'' issue, game UI does provide adequate information on character, game world, or enemies, visual indicators, icons, and maps. 
In this paper, we focus on glass game design, and we only explore new issues which do not exist in traditional game design.


\subsection{Head-Mounted Device Control}
Head gesture-based controls for head-mounted device is explored by some previous works. LoPresti et al. study the human ability to use head gesture-based control, they model the relationship between neck functional range and accuracy/speed in icon selection \cite{neck}.Hinkel et al. design a system that allows individuals to control a motorized wheelchair with simple head movements \cite{wheel}.
