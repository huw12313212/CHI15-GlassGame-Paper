\section{Related Work}

Although there are lack of offcial research focusing on glass game design, we can still borrow some experience from the research of existing Game Design and Head-Mounted Device Control.

\subsection{Game Design}

There are tons of research focusing on design guideline or heuristic rule for traditional video games \cite{gameflow,criticalreview,chi04game,09game,02game,08game,07game}. After the smart phone hitting the market, game design for mobile phone also become a hot research topic \cite{mobilegame,mobile06,mobile08,icec06},besides video and mobile game, there are only bits and pieces research on other specific platform. For example, Elena et al. focusing on using their unique gaming hardware ``Body Bug'' to explore body games\cite{bodygame}. Wetzel et al. explore the game design in augmented reaility\cite{argame},Mueller et al present a set of guidelines for movement-based game design\cite{movegame}. Bertelsmeyer et al provide an innovative wearable game and how to consider the characteristics of today's wearable hardware when developing a game\cite{wearable},but sadly they didn't consider smart glass as a gaming platform in their research. 
Although there are no research focusing on glass game design, some previous game design guideline are platform independent and we can use it directly. For example , Pinelle et al \cite{videogame} develop game heuristics in 10 categories, most of them are platform independent. Like ``Predictability'' issue , games should respond to users’ actions in a predictable manner. And ``Game status'' issue, gameing UI does provide adequate information on character, game world, or enemies. visual indicators, icons, and maps. In this paper, we are focusing on glass game design, and we only discuss the new issue which is not exist in traditional game design.


\subsection{Head-Mounted Device Control}
Head gesture-based controls for head-mounted device is explored by some previous works. LoPresti et al. study the human ability to use head gesture-based control, they model the relationship between neck functional range and accuracy/speed in icon selection \cite{neck}.Hinkel et al. design a system that allows individuals to control a motorized wheelchair with simple head movements \cite{wheel}.