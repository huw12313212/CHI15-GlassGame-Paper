\section{Related Work}

Although there is lack of offcial research focusing on glass game design, we still can borrow some experience from the research on existing game design and head-mounted device control.


\subsection{Game Design}

There are tons of research focusing on design guideline or heuristic rule for traditional video games\cite{gameflow,criticalreview,chi04game,09game,02game,08game,07game}. Nevertheless, after the smart phone hitting the market, game design for mobile phones also became a hot research topic\cite{mobilegame,mobile06,mobile08,icec06}. Besides video and mobile game, there are only bits and pieces research on other specific platform. For example, Elena et al. focused on using their unique gaming hardware, ``Body Bug'', to explore body games\cite{bodygame}; Wetzel et al. explored the game design in augmented reaility\cite{argame}; Mueller et al. presented a set of guidelines for movement-based game design\cite{movegame}. Bertelsmeyer et al. provided an innovative wearable game and researched in how to consider the characteristics of today's wearable hardware when developing a game\cite{wearable}. Unfortunately, they didn't consider the smart glass as a gaming platform in their research. 
From the previous research on related field, although there are no research focusing on glass game design, some previous game design guidelines are platform independent and we can use it directly. For instance, Pinelle et al.\cite{videogame} developed game heuristics in 10 categories, most of them are platform independent. 
Such as ``Predictability'' issue , games should respond to users' actions in a predictable manner. And ``Game status'' issue, game UI does provide adequate information on character, game world, or enemies, visual indicators, icons, and maps. 
In this paper, we focus on glass game design, and we only explore new issues which do not exist in traditional game design.


\subsection{Head Gesture-based Control}

Head gesture-based controls was explored by some previous works. LoPresti et al.\cite{neck} studied the human ability to use head gesture-based control. They modeled the relationship between neck functional range and accuracy/speed in icon selection; Hinkel et al.\cite{wheel} designed a system that allowed individuals to control a motorized wheelchair with simple head movements; Malkewitz et al.\cite{headdesktop} tried to map mouse and keyboard to speech and head motion, and enabled the possibility to add alternative input channels to common personal computers at moderate costs; Crossan et al.\cite{tilt} explored head tilting as an input technique to allow a user to interact with a mobile device ``hands free''; Zhu et al.\cite{tele} compared three different cameras' viewpoint control models and indicated the advantages of using the natural interaction model(combining eye gaze and head motion); Joseph et al.\cite{robot} explored the capabilities of head tracking combining with head mounted displays(HMD) as an input modality for robot navigation; Greuter et al.\cite{viewport} presented the use of commodity depth based cameras to control viewpoint from passive unencumbered head tracking. This allowed participants to maintain correct perspective while playing an immersive computer ball game.

Base on previous work, we can realize that head gesture-based control are really powerful for application. By using google glass, we can sense head gesture directly by glass gyro. However, how to use head gesture in glass game design is still an open question. In this work, we let users try different control style and explore the rule of using head gesture in glass game.



